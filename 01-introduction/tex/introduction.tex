%\documentclass[USenglish,pdftex,compress,10pt,svgnamesi,handout]{beamer}
\documentclass[USenglish,pdftex,compress,10pt,svgnamesi]{beamer}%


\usepackage{tikz}
\usetikzlibrary{positioning}
\usetikzlibrary{bayesnet}

\usefonttheme[onlymath]{serif}
\usepackage{graphicx}   % picture inclusion
\usepackage{amsmath}    % extended math stuff
\usepackage{amsfonts}   % corresponding fonts
\usepackage{ae}          % PDF-compatible fonts

\usepackage[english]{babel}


\setbeamertemplate{footline}[frame number]

\usetheme{Boadilla}
\usecolortheme[rgb={0,0.4,0}]{structure}
  \setbeamertemplate{enumerate items}[square]
  \setbeamertemplate{itemize items}[square]

\author{Patrick van der Smagt}
\hypersetup{%
	pdfauthor={Patrick van der Smagt}%
}

\newenvironment{poll}
{\begin{frame}<handout:0>
\frametitle{poll}
	\color[rgb]{0,.3,0}}
{\end{frame}}

\newcommand{\Def}[1]{\alert{#1}}                    % first occurrence of a new term
\newcommand{\Annot}[1]{\qquad\text{#1}}             % annotation in (short) equation
\newcommand*\T{^{\text{T}}}                         % transpose
\newcommand{\ParBeg}[1]{\structure{#1}}             % new paragraph, similar to description

\DeclareMathOperator*{\argmax}{arg\,max}
\DeclareMathOperator*{\argmin}{arg\,min}

\def\function{f}
\def\model{\hat\function}
\def\target{z}


\graphicspath{{pics/}}
\usepackage{picinpar}

\parskip2ex

\def\Vec#1{\textbf{#1}}
\def\bfx{{\Vec x}}
\def\bfw{{\Vec w}}
\def\cl#1{{\cal #1}}
\newcommand{\scalar}[2]{#1^T#2}
\newcommand{\scalarold}[2]{\left<#1,#2\right>}



% =====================================================================
% Titel etc.
\hypersetup{pdftitle={Introduction}}

\title{Introduction}
\date{March 2023}
% =====================================================================
\begin{document}

\begin{frame}
	\titlepage
\end{frame}


\begin{frame}
You have data from a stochastic system:
\begin{itemize}
  \item from an airport camera,
  \item from a camera seeing an insect flying, a body suit on a human, \dots
  \item weather observations, stock market data,
  \item a tactile sensor on a robot hand, etc.
\end{itemize}
and you want to use these data to
\begin{itemize}
  \item detect suspicious behaviour,
  \item classify movement,
  \item predict future data,
  \item control a system
\end{itemize}
How?
\end{frame}






\begin{frame}
The description is simple, in the form a model of your stochastic process:
$$
y = \function(x)
$$
where $y$, $x$ are vectors, and $\function()$ is the underlying model.

Consider the above also in the special form
$$
x_{t+1} = \function (x_t)
$$
where we use the Markov assumption.

Key question: how do we get $f()$?
\end{frame}



\begin{frame}
\frametitle{approximating $\function()$ through system knowledge}
If, e,g., we find all red objects in HSV space\\
 -- or we compute the Cartesian end position of a robot arm by inverse kinematics\\
 -- or we compute radiation of an exploding supernova\\
 -- or we want to determine a plant by analysing stem, leaves, flowers\\
 --\\
in each of these cases, we formalise expert knowledge in terms of a set of rules or decisions.

This is known classical control, expert systems, etc.

It is your best solution if your approximation (`model') $\model_\theta()$ to $\function()$ can be described at your desired accuracy.
The description of $\model()$ is obtained using innate knowledge of $\function$.
How do you know if your accuracy is good enough?
\end{frame}

\begin{frame}
\frametitle{approximating $f()$ when system knowledge fails}
If, e.g., you want to recognise faces, or a plant from a picture\\
 -- or want to create a programme to play go\\
 -- or want to find good grip positions on a random object\\
 -- or want to recognise a composer by listening to their music\\
 --\\
in each of these cases, the number of cases is too large to exhaustively describe accurately enough.

We have no detailed knowledge of $\function()$, and instead we use a general form for $\model()$ that can represent any function.
\end{frame}
\begin{frame}
But both approaches are in principle the same:
\begin{enumerate}
\item[1.] create a parameterised model $\model(x,\theta)$
\item[2.] find best values for the parameters $\theta$
\end{enumerate}

How do we do the second step?  Two options:\\
\begin{enumerate}
\item[a.] we make $\model$ behave as close as possible to $\function$ \\(e.g., maximise number of go games won)
\item[b.] we make $\model$ be as close as possible to $\function$ through
$
\min_\theta | \model_\theta(x) - \function(x) |.
$

Failing $\function$, we sample $\function$ in $\{x_i, y_i\}$ to 
$
\min_\theta \sum_i | \model_\theta(x_i) - y_i |
$
as a proxy.

We can write this as $\max_\theta \prod_i p_\theta (y_i \mid x_i)$.
\end{enumerate}

\end{frame}

\begin{frame}

The basic problem that we study in probability theory:
\begin{center}
Given a data generating process,\\ what are the properties of the outcomes?
\end{center}
The basic problem of statistics (or better statistical inference) \\is the inverse of probability theory:
\begin{center}
Given the outcomes,\\
what can we say about the process that generated the data?
\end{center}
Statistics uses the formal language of probability theory.

\end{frame}



\begin{frame}
\frametitle{discriminative vs.\ generative models}

The above describes \textit{discriminative} models, which map inputs to outputs and are described by $p(y|x)$.

Often we are interested in \textit{generative} models, represented by \\
$p(x, y)$ (if we have labels) or \\
$p(x)$ (if we have no labels). 


We can still measure quality, through $\max_\theta \prod_i p_\theta (x_i)$.

Generative models can be used to generate new data, or to test the likelihood of a datum.

\end{frame}



\begin{frame}
\frametitle{examples of discriminative models}

This is nothing new. Typically, such models use
$$\model_\theta(x) = \sum_k \theta_k s_k (x)$$
which describes, for instance, Fourier transform, polynomials, etc.

This general form is known as \textit{linear regression}, because the regression function is linear in the parameters $\theta$.

This is important as we have a closed-form solution to find $\theta$.

In the connectionism paradigm, the models can be drawn like this:

\includegraphics[scale=0.3]{nn0.pdf}
aka the perceptron
\end{frame}

\begin{frame}
\frametitle{curse of dimensionality}

Even if $x$ is only 100-dimensional

and you have a trillion data, 

\quad those data cover only $10^{-18}$ of the input space.\\[3ex]

This can be very problematic for such models.  

We need models that generalise further.
\end{frame}

\begin{frame}
\frametitle{extending the connectionist idea}
\includegraphics[scale=0.3]{nn1.pdf}
\qquad{\tiny did you see the small trick I did?}

That's a great idea from the 1970s which suggests,
$$\model_\theta(x) = \sum_k \theta_k s \left(\sum_i \theta_l x\right)$$
and the neural network (= multi-layer perceptron) is born.

Problem: finding $\theta$ can no longer be done closed-form.\\
Gradient-based optimisation is the standard approach.
\end{frame}


\begin{frame}
\frametitle{a deep neural network just has more layers}
\includegraphics[scale=0.3]{nn3.pdf}
\end{frame}



\frame{\frametitle{history of neural networks 1}

\begin{description}
\item[1960s:] the linear perceptron learns from data (Rosenblatt et al)
\item[1969:] the perceptron can't do XOR (Minsky \& Papert)
\item[1970--1980:] nonlinear networks trained with back-propagation  (Linnainmaa; Dreyfus; Werbos; Rumelhart)
\item[1990s:] one hidden layer can represent any (Borel-measurable) function
\item[mid 90s:] NN's can't do everything / do not generalise / \dots
\item[mid 90s:] support vector machines (SVMs) are great!
\item[1995--2000:] SVM too slow / too many SVs
\end{description}
}



\begin{frame}
\frametitle{history of neural networks 2}

\begin{description}
\item[2000--:] probabilistic models for machine learning (ML)

\item[2006:] \textbf{deep neural networks}, trained with \textsl{Restricted Boltzmann Machines} and backprop

\item[2009:] deep NNs can be trained with just BP,  \textbf{much compute power} (GPU) and \textbf{enough data}

\item[2011--:] recurrent neural networks resurrect for time-series modelling

\item[2012--:] convolutional neural networks (CNN) start winning most vision benchmarks; recurrent neural networks applied to speech recognition in Android

\item[2013--:] \textbf{probabilistic} NN (variance propagation; variational autoencoder)

\item[2015--:] show cases in robotics, sensory processing, \dots

\item[2017:] ``Attention is all you need'' \dots
\end{description}
\end{frame}


\begin{frame}
\frametitle{a recurrent neural network has an internal state}
\includegraphics[scale=0.3]{nn4.pdf}
\end{frame}
\begin{frame}
\frametitle{autoencoder: NN in a special form}
 \begin{tikzpicture}[overlay]
   \node[shift={(9cm,-35mm)}](a) {Dim(latent space) $\ll$ Dim($x$)};
\end{tikzpicture}

\includegraphics[scale=0.3]{ae.pdf}
\end{frame}
\begin{frame}
\frametitle{VAE:  probabilistic AE}
\begin{tikzpicture}[overlay]
   \node[shift={(9cm,-35mm)}](a) {`nonlinear PCA'};
\end{tikzpicture}

\includegraphics[scale=0.3]{vae.pdf}
\end{frame}




\begin{frame}
\frametitle{start experimenting}

Start the first notebook of this class on your machine

or use 
\url{https://colab.research.google.com/github/smagt/quick-intro-to-ml/}

\end{frame}


\begin{frame}
\frametitle{unsupervised learning: k-nearest neighbour}

 \includegraphics[width=\textwidth]{knn}

\end{frame}



\begin{frame}
\frametitle{1-NN algorithm}

Example: let's take $k=1$

\begin{enumerate}
\item define a distance measure
\item determine the number of classes
\item for each new data point $x$:
	\begin{enumerate}
	\item determine the distance to all other points
	\item find the nearest neighbours
	\end{enumerate}
\item the class $c$ of the new data point is determined
\end{enumerate}
\end{frame}



\begin{frame}
\frametitle{1-NN algorithm: overfitting}

 \includegraphics[width=\textwidth]{knn1}

\end{frame}



\begin{frame}
\frametitle{k-NN equation}

how can we describe this equation?

\pause

$$p(y = c \mid x) = {1 \over k} \sum_{i \in \mathrm{neighbours}} \delta_{y_ic}$$
mit
$$\delta_{ij} = \begin{cases}0 &\mathrm{if\ } i \neq j \\ 1 &\mathrm{if\ } i = j\end{cases}$$

\end{frame}



\begin{frame}
\frametitle{4-NN algorithm}

 \includegraphics[width=\textwidth]{knn5}

\end{frame}



\begin{frame}
\frametitle{finding hyperparameters}

$k$ is a hyperparameter, the value of which determines the outcome.
How can we optimise it?

\bigskip
 \includegraphics[scale=0.5]{xv0}

\pause
\bigskip
by cross validation

 \includegraphics[scale=0.5]{xv1}

\end{frame}



 
\begin{frame}
\frametitle{finding hyperparameters}
5-fold crossvalidation

 \includegraphics[scale=0.5]{xv2}

\end{frame}


\begin{frame}
\frametitle{finding hyperparameters}
but a more honest method is:
\begin{enumerate}
\item first learn\\
 \includegraphics[scale=0.5]{xv3}
 \bigskip
\item \pause then test on new data\\
 \includegraphics[scale=0.5]{xv4}
\end{enumerate}

\end{frame}



\begin{frame}
\frametitle{problems with kNN}

\begin{itemize}
\item finding the right distance measure is difficult and strongly influences the result
\item in high-dimensional spaces, distances stop making sense
{\tiny \url{https://stats.stackexchange.com/questions/99171/why-is-euclidean-distance-not-a-good-metric-in-high-dimensions}}
\item finding the neighbours is very  expensive in high-dimensional spaces
\item finding the neighbours is  very expensive if many data exist
\end{itemize}
\end{frame}


\end{document}
